\section{Lautstärkepegel}
\subsection{Lautstärkepegel}

\begin{lstlisting}
#include <stdint.h>
#include <stdbool.h>
#include "inc/hw_memmap.h"
#include "inc/hw_types.h"
#include "driverlib/sysctl.h"
#include "driverlib/adc.h"
#include "driverlib/gpio.h"
#include "driverlib/timer.h"

// Makros
#define FSAMPLE  44000
#define BUFFER_SIZE 1000

// globale Variable
int32_t buffer_sample[BUFFER_SIZE];     //Quadratische Signale
uint32_t i_sample = 0;
int32_t buffer_sample_sum = 0;          // momentaner Pegel
uint32_t index = 0;
uint32_t index_zuvor = 0;
// Prototypen
void ADC_int_handler(void);

int main(void)
{
    // SystemClock konfigurieren
    SysCtlClockSet(SYSCTL_SYSDIV_5|SYSCTL_USE_PLL|SYSCTL_OSC_MAIN|SYSCTL_XTAL_16MHZ);
    uint32_t ui32Period = SysCtlClockGet()/FSAMPLE;

    // Peripherie aktivieren
    SysCtlPeripheralEnable(SYSCTL_PERIPH_ADC0);
    SysCtlPeripheralEnable(SYSCTL_PERIPH_GPIOE);
    SysCtlPeripheralEnable(SYSCTL_PERIPH_GPIOB);
    SysCtlPeripheralEnable(SYSCTL_PERIPH_TIMER0);

    // GPIO konfigurieren
    GPIOPinTypeADC(GPIO_PORTE_BASE,GPIO_PIN_2);
    GPIOPinTypeGPIOOutput(GPIO_PORTB_BASE,GPIO_PIN_0|GPIO_PIN_1|GPIO_PIN_2|GPIO_PIN_3|GPIO_PIN_4|GPIO_PIN_5|GPIO_PIN_6|GPIO_PIN_7);

    //Timer0 konfigurieren
    TimerConfigure(TIMER0_BASE,TIMER_CFG_PERIODIC);
    TimerLoadSet(TIMER0_BASE, TIMER_A, ui32Period - 1);
    TimerControlTrigger(TIMER0_BASE,TIMER_A,true);
    TimerEnable(TIMER0_BASE,TIMER_A);

    // ADC konfigurieren
    ADCClockConfigSet(ADC0_BASE,ADC_CLOCK_RATE_FULL,1);

    ADCSequenceConfigure(ADC0_BASE, 3, ADC_TRIGGER_TIMER, 0);
    ADCSequenceStepConfigure(ADC0_BASE, 3, 0, ADC_CTL_CH1|ADC_CTL_IE|ADC_CTL_END);
    ADCSequenceEnable(ADC0_BASE, 3);
    ADCIntClear(ADC0_BASE,3);
    ADCIntRegister(ADC0_BASE,3,ADC_int_handler);
    ADCIntEnable(ADC0_BASE,3);

    while(1)
    {
    }
}
// Quadrierungsfunktion einer Variable x
int square(uint32_t x)                                          // Quadrierfunktion
{
   return x * x;
}
// Interrupt handler
void ADC_int_handler(void)
{
    ADCIntClear(ADC0_BASE, 3);  // delete interrupt flag
    ADCProcessorTrigger(ADC0_BASE,3);                            // Konvertierung beginnen

    while(!ADCIntStatus(ADC0_BASE,3,false))                      // warten bis Konvertierung abgeschlossen
    {
    }
    uint32_t NewADCValue = 0;
    ADCSequenceDataGet(ADC0_BASE,3,&NewADCValue);              // Wert einziehen
    NewADCValue = square(NewADCValue);                         //Wert quadrieren

    buffer_sample[index] = NewADCValue;                                 //speichert Werte im Vektor
    buffer_sample_sum += NewADCValue - buffer_sample[index_zuvor] ;     //Summe der gemessenen Werte berechnen
    index_zuvor= index;
    index++ ;

    if (index == BUFFER_SIZE )
        {
        index = 0;
        }
    //Grenzwerte bestimmt nach Test

    if(buffer_sample_sum < 50000)
            {
            GPIOPinWrite(GPIO_PORTB_BASE, GPIO_PIN_0| GPIO_PIN_1|GPIO_PIN_2|GPIO_PIN_3|GPIO_PIN_4|GPIO_PIN_5|GPIO_PIN_6| GPIO_PIN_7, 0b00000000);
            }
    if(buffer_sample_sum > 50000)
        {
        GPIOPinWrite(GPIO_PORTB_BASE, GPIO_PIN_0| GPIO_PIN_1|GPIO_PIN_2|GPIO_PIN_3|GPIO_PIN_4|GPIO_PIN_5|GPIO_PIN_6| GPIO_PIN_7, 0b00000001);
        }
    if(buffer_sample_sum > 100000)
        {
        GPIOPinWrite(GPIO_PORTB_BASE, GPIO_PIN_0| GPIO_PIN_1|GPIO_PIN_2|GPIO_PIN_3|GPIO_PIN_4|GPIO_PIN_5|GPIO_PIN_6| GPIO_PIN_7, 0b00000011);
        }
    if(buffer_sample_sum > 150000)
        {
        GPIOPinWrite(GPIO_PORTB_BASE, GPIO_PIN_0| GPIO_PIN_1|GPIO_PIN_2|GPIO_PIN_3|GPIO_PIN_4|GPIO_PIN_5|GPIO_PIN_6| GPIO_PIN_7, 0b00000111);
        }
    if(buffer_sample_sum > 200000)
        {
        GPIOPinWrite(GPIO_PORTB_BASE, GPIO_PIN_0| GPIO_PIN_1|GPIO_PIN_2|GPIO_PIN_3|GPIO_PIN_4|GPIO_PIN_5|GPIO_PIN_6| GPIO_PIN_7, 0b00001111);
        }
    if(buffer_sample_sum > 250000)
        {
        GPIOPinWrite(GPIO_PORTB_BASE, GPIO_PIN_0| GPIO_PIN_1|GPIO_PIN_2|GPIO_PIN_3|GPIO_PIN_4|GPIO_PIN_5|GPIO_PIN_6| GPIO_PIN_7, 0b00011111);
        }
    if(buffer_sample_sum > 300000)
        {
        GPIOPinWrite(GPIO_PORTB_BASE, GPIO_PIN_0| GPIO_PIN_1|GPIO_PIN_2|GPIO_PIN_3|GPIO_PIN_4|GPIO_PIN_5|GPIO_PIN_6| GPIO_PIN_7, 0b00111111);
        }
    if(buffer_sample_sum > 350000)
        {
        GPIOPinWrite(GPIO_PORTB_BASE, GPIO_PIN_0| GPIO_PIN_1|GPIO_PIN_2|GPIO_PIN_3|GPIO_PIN_4|GPIO_PIN_5|GPIO_PIN_6| GPIO_PIN_7, 0b01111111);
        }
    if(buffer_sample_sum > 400000)
        {
        GPIOPinWrite(GPIO_PORTB_BASE, GPIO_PIN_0| GPIO_PIN_1|GPIO_PIN_2|GPIO_PIN_3|GPIO_PIN_4|GPIO_PIN_5|GPIO_PIN_6| GPIO_PIN_7, 0b11111111);
        }
    }
\end{lstlisting}

\subsection{Eigene Verbesserungen}
Unser Vorschlag zu einer Verbesserung des bisherigen Codes besteht daraus den Spannungspegel einer Solarzelle auf Knopfdruck zu messen und mittels der LEDs wiederzugeben. \\ 
Um dies umzusetzen, wird zuerst die Spannung mit dem Pin PE2 gemessen und der Code von der vorherigen Teilaufgabe benutzt. Es wird eine Zener-Diode verwendet um das Board zu schützen.  Bei maximaler Beleuchtung gingen die LEDs aus. Unsere Idee war, dass die gemessene Spannung ein Overflow erzeugt, also wurde ein \SI{1}{\mega \ohm} Widerstand zwischen der Solarzelle und dem Pin eingebaut. Nach dieser Veränderung blieben die LEDs an bei maximaler beleuchtung. Bei geringer Spannung schlagen die LEDs schnell aus, also mussten die A/D Grenzen verändert werden. Nach einigen Versuchen wurde der maximale Pegel auf einen A/D-Wert von 2400000 gesetzt und für die anderen LEDs wurden in Werte in 300000 Schritten reduziert. Es wäre aber eine Energieverschwendung konstant den Spannungspegel zu messen. Deshalb wird zusätzlich ein Knopf aktiviert, welcher beim drücken den Spannungspegel wiedergibt.

\begin{lstlisting}
#include <stdint.h>
#include <stdbool.h>
#include "inc/hw_memmap.h"
#include "inc/hw_types.h"
#include "driverlib/sysctl.h"
#include "driverlib/adc.h"
#include "driverlib/gpio.h"
#include "driverlib/timer.h"

// Makros
#define FSAMPLE  44000
#define BUFFER_SIZE 1000

// globale Variable
int32_t buffer_sample[BUFFER_SIZE];     //Quadratische Signale
uint32_t i_sample = 0;
int32_t buffer_sample_sum = 0;          // momentaner Pegel
uint32_t index = 0;
uint32_t index_zuvor = 0;
// Prototypen
void ADC_int_handler(void);

int main(void)
{
    // SystemClock konfigurieren
    SysCtlClockSet(SYSCTL_SYSDIV_5|SYSCTL_USE_PLL|SYSCTL_OSC_MAIN|SYSCTL_XTAL_16MHZ);
    uint32_t ui32Period = SysCtlClockGet()/FSAMPLE;

    // Peripherie aktivieren
    SysCtlPeripheralEnable(SYSCTL_PERIPH_ADC0);
    SysCtlPeripheralEnable(SYSCTL_PERIPH_GPIOE);
    SysCtlPeripheralEnable(SYSCTL_PERIPH_GPIOB);
    SysCtlPeripheralEnable(SYSCTL_PERIPH_TIMER0);

    // GPIO konfigurieren
    GPIOPinTypeADC(GPIO_PORTE_BASE,GPIO_PIN_2);
    GPIOPinTypeGPIOOutput(GPIO_PORTB_BASE,GPIO_PIN_0|GPIO_PIN_1|GPIO_PIN_2|GPIO_PIN_3|GPIO_PIN_4|GPIO_PIN_5|GPIO_PIN_6|GPIO_PIN_7);

    //Timer0 konfigurieren
    TimerConfigure(TIMER0_BASE,TIMER_CFG_PERIODIC);
    TimerLoadSet(TIMER0_BASE, TIMER_A, ui32Period - 1);
    TimerControlTrigger(TIMER0_BASE,TIMER_A,true);
    TimerEnable(TIMER0_BASE,TIMER_A);

    // ADC konfigurieren
    ADCClockConfigSet(ADC0_BASE,ADC_CLOCK_RATE_FULL,1);

    ADCSequenceConfigure(ADC0_BASE, 3, ADC_TRIGGER_TIMER, 0);
    ADCSequenceStepConfigure(ADC0_BASE, 3, 0, ADC_CTL_CH1|ADC_CTL_IE|ADC_CTL_END);
    ADCSequenceEnable(ADC0_BASE, 3);
    ADCIntClear(ADC0_BASE,3);
    ADCIntRegister(ADC0_BASE,3,ADC_int_handler);
    ADCIntEnable(ADC0_BASE,3);

    //Schalter Config
    SysCtlPeripheralEnable(SYSCTL_PERIPH_GPIOF);                // Port F Aktivieren
    GPIOPinTypeGPIOInput(GPIO_PORTF_BASE,GPIO_PIN_4);           // PF4 als Input definieren
    GPIOPadConfigSet(GPIO_PORTF_BASE,GPIO_PIN_4,GPIO_STRENGTH_2MA,GPIO_PIN_TYPE_STD_WPU);

    while(1)
    {
    }
}
// Quadrierungsfunktion einer Variable x
int square(uint32_t x)                                          // Quadrierfunktion
{
   return x * x;
}
// Interrupt handler
void ADC_int_handler(void)
{
    ADCIntClear(ADC0_BASE, 3);  // delete interrupt flag
    ADCProcessorTrigger(ADC0_BASE,3);                            // Konvertierung beginnen

    while(!ADCIntStatus(ADC0_BASE,3,false))                      // warten bis Konvertierung abgeschlossen
    {
    }
    uint32_t NewADCValue = 0;
    ADCSequenceDataGet(ADC0_BASE,3,&NewADCValue);              // Wert einziehen
    NewADCValue = square(NewADCValue);                         //Wert quadrieren

    buffer_sample[index] = NewADCValue;                                 //speichert Werte im Vektor
    buffer_sample_sum += NewADCValue - buffer_sample[index_zuvor] ;     //Summe der gemessenen Werte berechnen
    index_zuvor= index;
    index++ ;

    if (index == BUFFER_SIZE )
        {
        index = 0;
        }
    //Grenzwerte bestimmt nach Test
    if(GPIOPinRead(GPIO_PORTF_BASE,GPIO_PIN_4)==0) {

    if(buffer_sample_sum < 300000)
            {
            GPIOPinWrite(GPIO_PORTB_BASE, GPIO_PIN_0| GPIO_PIN_1|GPIO_PIN_2|GPIO_PIN_3|GPIO_PIN_4|GPIO_PIN_5|GPIO_PIN_6| GPIO_PIN_7, 0b00000000);
            }
    if(buffer_sample_sum > 300000)
        {
        GPIOPinWrite(GPIO_PORTB_BASE, GPIO_PIN_0| GPIO_PIN_1|GPIO_PIN_2|GPIO_PIN_3|GPIO_PIN_4|GPIO_PIN_5|GPIO_PIN_6| GPIO_PIN_7, 0b00000001);
        }
    if(buffer_sample_sum > 600000)
        {
        GPIOPinWrite(GPIO_PORTB_BASE, GPIO_PIN_0| GPIO_PIN_1|GPIO_PIN_2|GPIO_PIN_3|GPIO_PIN_4|GPIO_PIN_5|GPIO_PIN_6| GPIO_PIN_7, 0b00000011);
        }
    if(buffer_sample_sum > 900000)
        {
        GPIOPinWrite(GPIO_PORTB_BASE, GPIO_PIN_0| GPIO_PIN_1|GPIO_PIN_2|GPIO_PIN_3|GPIO_PIN_4|GPIO_PIN_5|GPIO_PIN_6| GPIO_PIN_7, 0b00000111);
        }
    if(buffer_sample_sum > 1200000)
        {
        GPIOPinWrite(GPIO_PORTB_BASE, GPIO_PIN_0| GPIO_PIN_1|GPIO_PIN_2|GPIO_PIN_3|GPIO_PIN_4|GPIO_PIN_5|GPIO_PIN_6| GPIO_PIN_7, 0b00001111);
        }
    if(buffer_sample_sum > 1500000)
        {
        GPIOPinWrite(GPIO_PORTB_BASE, GPIO_PIN_0| GPIO_PIN_1|GPIO_PIN_2|GPIO_PIN_3|GPIO_PIN_4|GPIO_PIN_5|GPIO_PIN_6| GPIO_PIN_7, 0b00011111);
        }
    if(buffer_sample_sum > 1800000)
        {
        GPIOPinWrite(GPIO_PORTB_BASE, GPIO_PIN_0| GPIO_PIN_1|GPIO_PIN_2|GPIO_PIN_3|GPIO_PIN_4|GPIO_PIN_5|GPIO_PIN_6| GPIO_PIN_7, 0b00111111);
        }
    if(buffer_sample_sum > 2100000)
        {
        GPIOPinWrite(GPIO_PORTB_BASE, GPIO_PIN_0| GPIO_PIN_1|GPIO_PIN_2|GPIO_PIN_3|GPIO_PIN_4|GPIO_PIN_5|GPIO_PIN_6| GPIO_PIN_7, 0b01111111);
        }
    if(buffer_sample_sum > 2400000)
        {
        GPIOPinWrite(GPIO_PORTB_BASE, GPIO_PIN_0| GPIO_PIN_1|GPIO_PIN_2|GPIO_PIN_3|GPIO_PIN_4|GPIO_PIN_5|GPIO_PIN_6| GPIO_PIN_7, 0b11111111);
        }
    }
    else
    {
        GPIOPinWrite(GPIO_PORTB_BASE, GPIO_PIN_0| GPIO_PIN_1|GPIO_PIN_2|GPIO_PIN_3|GPIO_PIN_4|GPIO_PIN_5|GPIO_PIN_6| GPIO_PIN_7, 0b00000000);
    }
}
\end{lstlisting}
