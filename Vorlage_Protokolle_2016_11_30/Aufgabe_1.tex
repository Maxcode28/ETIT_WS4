\section{Interrupts}
Ein Interrupt im Allgemeinen ist eine Unterbrechung eines laufenden Prozesses um zeitkritische Prozesse direkt abarbeiten zu können. Sie sind nicht Bestandteil des laufenden Programmes, sondern werden durch asynchrone, externe Ereignisse ausgelöst und benötigen eine extra Hardware. So wird beispielsweise bei einem Druck einer Taste einer Tastatur ein Interrupt ausgelöst um den jeweiligen Befehl direkt ausführen zu können. Eine Sonderklasse der Interrupts sind die non maskable Interrupts, kurz NMI. Diese sind höhergestellt als normale Interrupts und werden mit höchster Priorität direkt abgearbeitet und können nicht von dem Prozessor abgelehnt werden. Im Falle eines Stromausfalles greifen diese ein und sichern wichtige Daten bevor es zum Absturz kommt. Grundsätzlich Unterscheidet man Interrupts weiter in Hardware, Software, präzise und unpräzise Interrupts. Hardware Interrupts sind hierbei Interrupts die von einer angeschlossenen Hardware an die CPU weitergeleitet werden. Dem entgegen stehen die Software Interrupts, welche wie ein Hardware Interrupt wirken, jedoch von einem Programm aufgerufen werden. Der Ablauf von Interruptzyklen sieht folgender maßen aus: Wenn kein NMI vorliegt wird bei einem Interrupt bis Ende des Befehls gewartet. Nach abarbeiten des Befehls wird der Interruptzyklus der CPU gestartet. Hierbei liest der Datenbus den Interruptvektor und sperrt den maskierten Interrupteingang um ein überschneiden mehrerer Interrupts zu verhindern. Um im Anschluss wieder nahtlos das Programm fortführen zu können speichert die CPU den Befehlszähler im Stack ab. Mit Hilfe des Interruptvektors und einer Interrupttabelle kann der Interruptzeiger bestimmt werden. Anschließend läuft die Interrupt-Service-Routine ab. Hierbei ist zu beachten, dass alle Daten des vorherigen Programms zuvor in den Stack kopiert werden und nach Ablauf des Programms ebenfalls wieder zurück kopiert werden.