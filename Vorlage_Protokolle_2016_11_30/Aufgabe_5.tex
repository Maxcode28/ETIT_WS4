\section{Warteschleifen und Taktfrequenz}
\subsection{}
Der Mikrokontroller besitzt einen 16 MHz Quarz, welcher mit Hilfe der PLL 400 MHz erreicht. Durch den Divisor Div\_ 5 wird die gegebene Frequenz noch durch 5 dividiert wird. Somit ergibt sich mit der internen Halbierung eine Frequenz von 40 MHz. Die Berechnung ist in \autoref{eq:5.1} gezeigt.
 
\begin{equation}
\label{eq:5.1}
\frac{400\,\text{MHz}}{2\times 5}=40\,\text{MHz}
\end{equation}

\subsection{}
\begin{lstlisting}
#include <stdint.h>
#include <stdbool.h>
#include "inc/hw_ints.h"
#include "inc/hw_memmap.h"
#include "inc/hw_types.h"
#include "driverlib/gpio.h"
#include "driverlib/sysctl.h"
#include "driverlib/timer.h"
#include "driverlib/interrupt.h"

// stores ms since startup
volatile uint32_t systemTime_ms = 0;

void InterruptHandlerTimer0A (void)
{
    // Clear the timer interrupt to prevent the interrupt function from immediately being called again on exit
    TimerIntClear(TIMER0_BASE, TIMER_TIMA_TIMEOUT);
    // count up one ms
    systemTime_ms++;
}

void clockSetup(void)
{
    uint32_t timerPeriod;
    //Configure clock
    SysCtlClockSet(SYSCTL_SYSDIV_5|SYSCTL_USE_PLL|SYSCTL_XTAL_16MHZ|SYSCTL_OSC_MAIN);
    //enable peripheral for timer
    SysCtlPeripheralEnable(SYSCTL_PERIPH_TIMER0);
    //configure timer as 32 bit timer in periodic mode
    TimerConfigure(TIMER0_BASE, TIMER_CFG_PERIODIC);
    //set timerPeriod to number of periods needed to generate a timeout with a frequency of 1kHz (every 1ms)
    timerPeriod = (SysCtlClockGet()/1000);
    //set TIMER-0-A to generate a timeout after timerPeriod-1 cycles
    TimerLoadSet(TIMER0_BASE, TIMER_A, timerPeriod-1);
    //Register the function InterruptHandlerTimer0A to be called when an interrupt from TIMER-0-A occurs
    TimerIntRegister(TIMER0_BASE, TIMER_A, &(InterruptHandlerTimer0A));
    //Enable the interrupt for TIMER-0-A
    IntEnable(INT_TIMER0A);
    //generate an interrupt, when TIMER-0-A sends a timeout
    TimerIntEnable(TIMER0_BASE, TIMER_TIMA_TIMEOUT);
    //master interrupt enable for all interrupts
    IntMasterEnable();
    //Enable the timer to start counting
    TimerEnable(TIMER0_BASE, TIMER_A);
}

void delay_ms(uint32_t waitTime)
{
    uint32_t t = systemTime_ms;
    while (systemTime_ms - t < waitTime);
}

int main(void)
{
clockSetup();
        SysCtlPeripheralEnable(SYSCTL_PERIPH_GPIOB);               // GPIO Port B aktivieren
            GPIOPinTypeGPIOOutput(GPIO_PORTB_BASE, GPIO_PIN_0);         // GPIO Port B Pin 0-7 aktivieren
            GPIOPinTypeGPIOOutput(GPIO_PORTB_BASE, GPIO_PIN_1);
            GPIOPinTypeGPIOOutput(GPIO_PORTB_BASE, GPIO_PIN_2);
            GPIOPinTypeGPIOOutput(GPIO_PORTB_BASE, GPIO_PIN_3);
            GPIOPinTypeGPIOOutput(GPIO_PORTB_BASE, GPIO_PIN_4);
            GPIOPinTypeGPIOOutput(GPIO_PORTB_BASE, GPIO_PIN_5);
            GPIOPinTypeGPIOOutput(GPIO_PORTB_BASE, GPIO_PIN_6);
            GPIOPinTypeGPIOOutput(GPIO_PORTB_BASE, GPIO_PIN_7);

            bool state = 0;
            //Verbeseserte Variante des Codes der GPIO-Steuerung aus Aufgabe 4
                while (1)
                {
                    GPIOPinWrite(GPIO_PORTB_BASE, GPIO_PIN_0, GPIO_PIN_0*state);
                    GPIOPinWrite(GPIO_PORTB_BASE, GPIO_PIN_1, GPIO_PIN_1*state);
                    GPIOPinWrite(GPIO_PORTB_BASE, GPIO_PIN_2, GPIO_PIN_2*state);
                    GPIOPinWrite(GPIO_PORTB_BASE, GPIO_PIN_3, GPIO_PIN_3*state);
                    GPIOPinWrite(GPIO_PORTB_BASE, GPIO_PIN_4, GPIO_PIN_4*state);
                    GPIOPinWrite(GPIO_PORTB_BASE, GPIO_PIN_5, GPIO_PIN_5*state);
                    GPIOPinWrite(GPIO_PORTB_BASE, GPIO_PIN_6, GPIO_PIN_6*state);
                    GPIOPinWrite(GPIO_PORTB_BASE, GPIO_PIN_7, GPIO_PIN_7*state);

                    state = !state;
                    delay_ms(500/2);
                }
            }
\end{lstlisting}
\subsection{}
\begin{table}[htb]
    \centering
    \caption{Flimmerverschmelzungsfrequenz}
    \label{tab:Flimmerverschmelzungsfrequenz}
    \begin{tabular}{cc}
        \toprule
        Name & Frequenz [\si{\hertz}] \\
        \midrule
        David & \num{50}  \\
        Jan & \num{47.6} \\
        Max & \num{45} \\
        \bottomrule
    \end{tabular}
    
\end{table}