\section{Warteschleifen und Taktfrequenz}
\subsection{}
Der Mikrokontroller besitzt einen 16 MHz Quarz, welcher mit Hilfe der PLL 400 MHz erreicht. Durch den Divisor Div\_ 5 wird die gegebene Frequenz noch durch 5 dividiert wird. Somit ergibt sich mit der internen Halbierung eine Frequenz von 40 MHz. Die Berechnung ist in \autoref{eq:5.1} gezeigt.
 
\begin{equation}
\label{eq:5.1}
\frac{400\,\text{MHz}}{2\times 5}=40\,\text{MHz}
\end{equation}

\subsection{}
Siehe Protokoll.

\subsection{}
\begin{table}[htb]
    \centering
    \caption{Flimmerverschmelzungsfrequenz}
    \label{tab:Flimmerverschmelzungsfrequenz}
    \begin{tabular}{cc}
        \toprule
        Name & Frequenz [\si{\hertz}] \\
        \midrule
        David & \num{50}  \\
        Jan & \num{47.6} \\
        Max & \num{45} \\
        \bottomrule
    \end{tabular}
    
\end{table}