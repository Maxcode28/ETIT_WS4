\section{Digitale Eingänge}


/**
 * main.c
 */
 #include <stdint.h>
 #include <stdbool.h>                                           // definition of type "bool"
 #include"inc/hw_memmap.h"                                      // definition of memory adresses
 #include"inc/hw_types.h"                                        // definition of framework makros
 #include"driverlib/gpio.h"
 #include"driverlib/sysctl.h"

void delay(void)                                                // Blinklicht Delay
{
  uint32_t i=500000;
  while(i) {i--;}
 }

void delayPWM(void)                                             // PWM Delay (Hoher Wert -> Helle LED)
{
  uint32_t i=10000;
  while(i) {i--;}
 }


int main(void){
    SysCtlPeripheralEnable(SYSCTL_PERIPH_GPIOB);                // GPIO Port B aktivieren
    GPIOPinTypeGPIOOutput(GPIO_PORTB_BASE, GPIO_PIN_0);         // GPIO Port B Pin 0-7 aktivieren
    GPIOPinTypeGPIOOutput(GPIO_PORTB_BASE, GPIO_PIN_1);
    GPIOPinTypeGPIOOutput(GPIO_PORTB_BASE, GPIO_PIN_2);
    GPIOPinTypeGPIOOutput(GPIO_PORTB_BASE, GPIO_PIN_3);
    GPIOPinTypeGPIOOutput(GPIO_PORTB_BASE, GPIO_PIN_4);
    GPIOPinTypeGPIOOutput(GPIO_PORTB_BASE, GPIO_PIN_5);
    GPIOPinTypeGPIOOutput(GPIO_PORTB_BASE, GPIO_PIN_6);
    GPIOPinTypeGPIOOutput(GPIO_PORTB_BASE, GPIO_PIN_7);

    GPIOPinTypeGPIOInput(GPIO_PORTC_BASE,GPIO_PIN_7);           // Schalter input config
    GPIOPadConfigSet(GPIO_PORTC_BASE,GPIO_PIN_7,GPIO_STRENGTH_2MA, GPIO_PIN_TYPE_STD_WPU);

   while(1)
   {
       delay();
       if(GPIOPinRead(GPIO_PORTC_BASE,GPIO_PIN_7)==0) {
    GPIOPinWrite(GPIO_PORTB_BASE, GPIO_PIN_0, 0xFF);            // GPIO Pins 0-7 auf High stellen
    GPIOPinWrite(GPIO_PORTB_BASE, GPIO_PIN_1, 0xFF);
    GPIOPinWrite(GPIO_PORTB_BASE, GPIO_PIN_2, 0xFF);
    GPIOPinWrite(GPIO_PORTB_BASE, GPIO_PIN_3, 0xFF);
    GPIOPinWrite(GPIO_PORTB_BASE, GPIO_PIN_4, 0xFF);
    GPIOPinWrite(GPIO_PORTB_BASE, GPIO_PIN_5, 0xFF);
    GPIOPinWrite(GPIO_PORTB_BASE, GPIO_PIN_6, 0xFF);
    GPIOPinWrite(GPIO_PORTB_BASE, GPIO_PIN_7, 0xFF);
       }
       else
       {
    GPIOPinWrite(GPIO_PORTB_BASE, GPIO_PIN_0, 0x00);            // GPIO Pins 0-7 auf Low stellen
    GPIOPinWrite(GPIO_PORTB_BASE, GPIO_PIN_1, 0x00);
    GPIOPinWrite(GPIO_PORTB_BASE, GPIO_PIN_2, 0x00);
    GPIOPinWrite(GPIO_PORTB_BASE, GPIO_PIN_3, 0x00);
    GPIOPinWrite(GPIO_PORTB_BASE, GPIO_PIN_4, 0x00);
    GPIOPinWrite(GPIO_PORTB_BASE, GPIO_PIN_5, 0x00);
    GPIOPinWrite(GPIO_PORTB_BASE, GPIO_PIN_6, 0x00);
    GPIOPinWrite(GPIO_PORTB_BASE, GPIO_PIN_7, 0x00);
       }
   }
    return 0;
}

\subsection{Taster}

/**
 * main.c
 */
 #include <stdint.h>
 #include <stdbool.h>                                           // definition of type "bool"
 #include"inc/hw_memmap.h"                                      // definition of memory adresses
 #include"inc/hw_types.h"                                        // definition of framework makros
 #include"driverlib/gpio.h"
 #include"driverlib/sysctl.h"

void delay(void)
{
  uint32_t i=200000;
  while(i) {i--;}
 }


int main(void){

    SysCtlPeripheralEnable(SYSCTL_PERIPH_GPIOF);                // Port F Aktivieren
    GPIOPinTypeGPIOInput(GPIO_PORTF_BASE,GPIO_PIN_4);           // PF4 als Input definieren
    GPIOPadConfigSet(GPIO_PORTF_BASE,GPIO_PIN_4,GPIO_STRENGTH_2MA,GPIO_PIN_TYPE_STD_WPU);
    // Bei der der Unteraufagbe 6.1.1 & 6.1.2 müsste der der Port B und Pin 7 gewählt werden

    SysCtlPeripheralEnable(SYSCTL_PERIPH_GPIOB);                // GPIO Port B aktivieren
    GPIOPinTypeGPIOOutput(GPIO_PORTB_BASE, GPIO_PIN_0);         // GPIO Port B Pin 0-7 aktivieren
    GPIOPinTypeGPIOOutput(GPIO_PORTB_BASE, GPIO_PIN_1);
    GPIOPinTypeGPIOOutput(GPIO_PORTB_BASE, GPIO_PIN_2);
    GPIOPinTypeGPIOOutput(GPIO_PORTB_BASE, GPIO_PIN_3);
    GPIOPinTypeGPIOOutput(GPIO_PORTB_BASE, GPIO_PIN_4);
    GPIOPinTypeGPIOOutput(GPIO_PORTB_BASE, GPIO_PIN_5);
    GPIOPinTypeGPIOOutput(GPIO_PORTB_BASE, GPIO_PIN_6);
    GPIOPinTypeGPIOOutput(GPIO_PORTB_BASE, GPIO_PIN_7);

    bool state = 0;

   while(1)
   {
       delay();

       if(GPIOPinRead(GPIO_PORTF_BASE,GPIO_PIN_4)==0)               // Wenn Schalter geschlossen
       {
          state = !state;                                           // State wird gewechselt
       }

       GPIOPinWrite(GPIO_PORTB_BASE, GPIO_PIN_0, GPIO_PIN_0*state); // GPIO Output wird auf jeweiligen State geschaltet
       GPIOPinWrite(GPIO_PORTB_BASE, GPIO_PIN_1, GPIO_PIN_1*state);
       GPIOPinWrite(GPIO_PORTB_BASE, GPIO_PIN_2, GPIO_PIN_2*state);
       GPIOPinWrite(GPIO_PORTB_BASE, GPIO_PIN_3, GPIO_PIN_3*state);
       GPIOPinWrite(GPIO_PORTB_BASE, GPIO_PIN_4, GPIO_PIN_4*state);
       GPIOPinWrite(GPIO_PORTB_BASE, GPIO_PIN_5, GPIO_PIN_5*state);
       GPIOPinWrite(GPIO_PORTB_BASE, GPIO_PIN_6, GPIO_PIN_6*state);
       GPIOPinWrite(GPIO_PORTB_BASE, GPIO_PIN_7, GPIO_PIN_7*state);
   }
    return 0;
}

\subsection{Externer Interrupt}
