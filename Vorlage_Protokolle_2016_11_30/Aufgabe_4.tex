\section{Verwenden des TivaWare-Framework}
\subsection{GPIO-Blinklicht}
\begin{lstlisting}
#include <stdint.h>
 #include <stdbool.h> // definition of type "bool"
 #include"inc/hw_memmap.h" // definition of memory adresses
 #include"inc/hw_types.h" // definition of framework makros
 #include"driverlib/gpio.h"
 #include"driverlib/sysctl.h"
 
// Delay-Funktion

void delay(void)
{
  uint32_t i=80000;
  while(i) {i--;}
 }


int main(void){
    SysCtlPeripheralEnable(SYSCTL_PERIPH_GPIOF);        //Port F aktivieren
    GPIOPinTypeGPIOOutput(GPIO_PORTF_BASE, GPIO_PIN_1);   //Pin 1 bis 3 initialisieren
    GPIOPinTypeGPIOOutput(GPIO_PORTF_BASE, GPIO_PIN_2);
    GPIOPinTypeGPIOOutput(GPIO_PORTF_BASE, GPIO_PIN_3);

   while(1)      //unendliche Schleife fuer flimmern der LED in weiss
   {
       delay();
    GPIOPinWrite(GPIO_PORTF_BASE, GPIO_PIN_1, 0xFF);
    GPIOPinWrite(GPIO_PORTF_BASE, GPIO_PIN_2, 0xFF);
    GPIOPinWrite(GPIO_PORTF_BASE, GPIO_PIN_3, 0xFF);
        delay();
    GPIOPinWrite(GPIO_PORTF_BASE, GPIO_PIN_1, 0x00);
    GPIOPinWrite(GPIO_PORTF_BASE, GPIO_PIN_2, 0x00);
    GPIOPinWrite(GPIO_PORTF_BASE, GPIO_PIN_3, 0x00);


   }
    return 0;
}
\end{lstlisting}




\subsection{Externe LED Schaltung}
\subsubsection{}
\begin{lstlisting}
/**
 * main.c
 */
 #include <stdint.h>
 #include <stdbool.h>                                           // definition of type "bool"
 #include"inc/hw_memmap.h"                                      // definition of memory adresses
#include"inc/hw_types.h"                                        // definition of framework makros
 #include"driverlib/gpio.h"
 #include"driverlib/sysctl.h"

void delay(void)
{
  uint32_t i=100000;
  while(i) {i--;}
 }


int main(void){
    SysCtlPeripheralEnable(SYSCTL_PERIPH_GPIOB);                // GPIO Port B aktivieren
    GPIOPinTypeGPIOOutput(GPIO_PORTB_BASE, GPIO_PIN_0);         // GPIO Port B Pin 0-7 aktivieren
    GPIOPinTypeGPIOOutput(GPIO_PORTB_BASE, GPIO_PIN_1);
    GPIOPinTypeGPIOOutput(GPIO_PORTB_BASE, GPIO_PIN_2);
    GPIOPinTypeGPIOOutput(GPIO_PORTB_BASE, GPIO_PIN_3);
    GPIOPinTypeGPIOOutput(GPIO_PORTB_BASE, GPIO_PIN_4);
    GPIOPinTypeGPIOOutput(GPIO_PORTB_BASE, GPIO_PIN_5);
    GPIOPinTypeGPIOOutput(GPIO_PORTB_BASE, GPIO_PIN_6);
    GPIOPinTypeGPIOOutput(GPIO_PORTB_BASE, GPIO_PIN_7);

   while(1)
   {
       delay();
    GPIOPinWrite(GPIO_PORTB_BASE, GPIO_PIN_0, 0xFF);            // GPIO Pins auf High stellen
    GPIOPinWrite(GPIO_PORTB_BASE, GPIO_PIN_1, 0xFF);
    GPIOPinWrite(GPIO_PORTB_BASE, GPIO_PIN_2, 0xFF);
    GPIOPinWrite(GPIO_PORTB_BASE, GPIO_PIN_3, 0xFF);
    GPIOPinWrite(GPIO_PORTB_BASE, GPIO_PIN_4, 0xFF);
    GPIOPinWrite(GPIO_PORTB_BASE, GPIO_PIN_5, 0xFF);
    GPIOPinWrite(GPIO_PORTB_BASE, GPIO_PIN_6, 0xFF);
    GPIOPinWrite(GPIO_PORTB_BASE, GPIO_PIN_7, 0xFF);
        delay();
    GPIOPinWrite(GPIO_PORTB_BASE, GPIO_PIN_0, 0x00);            // GPIO Pins auf Low stellen
    GPIOPinWrite(GPIO_PORTB_BASE, GPIO_PIN_1, 0x00);
    GPIOPinWrite(GPIO_PORTB_BASE, GPIO_PIN_2, 0x00);
    GPIOPinWrite(GPIO_PORTB_BASE, GPIO_PIN_3, 0x00);
    GPIOPinWrite(GPIO_PORTB_BASE, GPIO_PIN_4, 0x00);
    GPIOPinWrite(GPIO_PORTB_BASE, GPIO_PIN_5, 0x00);
    GPIOPinWrite(GPIO_PORTB_BASE, GPIO_PIN_6, 0x00);
    GPIOPinWrite(GPIO_PORTB_BASE, GPIO_PIN_7, 0x00);

   }
    return 0;
}
\end{lstlisting}

\subsubsection{}

\begin{lstlisting}

/**
 * main.c
 */
 #include <stdint.h>
 #include <stdbool.h>                                           // definition of type "bool"
 #include"inc/hw_memmap.h"                                      // definition of memory adresses
#include"inc/hw_types.h"                                        // definition of framework makros
 #include"driverlib/gpio.h"
 #include"driverlib/sysctl.h"

void delay(void)                                                // Blinklicht Delay
{
  uint32_t i=1000;                                              
  while(i) {i--;}
 }

void delayPWM(void)                                             // PWM Delay (Hoher Wert -> Helle LED)
{
  uint32_t i=8000;                                            
  while(i) {i--;}
 }


int main(void){
    SysCtlPeripheralEnable(SYSCTL_PERIPH_GPIOB);                // GPIO Port B aktivieren
    GPIOPinTypeGPIOOutput(GPIO_PORTB_BASE, GPIO_PIN_0);         // GPIO Port B Pin 0-7 aktivieren
    GPIOPinTypeGPIOOutput(GPIO_PORTB_BASE, GPIO_PIN_1);
    GPIOPinTypeGPIOOutput(GPIO_PORTB_BASE, GPIO_PIN_2);
    GPIOPinTypeGPIOOutput(GPIO_PORTB_BASE, GPIO_PIN_3);
    GPIOPinTypeGPIOOutput(GPIO_PORTB_BASE, GPIO_PIN_4);
    GPIOPinTypeGPIOOutput(GPIO_PORTB_BASE, GPIO_PIN_5);
    GPIOPinTypeGPIOOutput(GPIO_PORTB_BASE, GPIO_PIN_6);
    GPIOPinTypeGPIOOutput(GPIO_PORTB_BASE, GPIO_PIN_7);

   while(1)
   {
       delay();
       
    GPIOPinWrite(GPIO_PORTB_BASE, GPIO_PIN_0, 0xFF);            // GPIO Pins 0-7 auf High stellen
    GPIOPinWrite(GPIO_PORTB_BASE, GPIO_PIN_1, 0xFF);
    GPIOPinWrite(GPIO_PORTB_BASE, GPIO_PIN_2, 0xFF);
    GPIOPinWrite(GPIO_PORTB_BASE, GPIO_PIN_3, 0xFF);
    GPIOPinWrite(GPIO_PORTB_BASE, GPIO_PIN_4, 0xFF);
    GPIOPinWrite(GPIO_PORTB_BASE, GPIO_PIN_5, 0xFF);
    GPIOPinWrite(GPIO_PORTB_BASE, GPIO_PIN_6, 0xFF);
    GPIOPinWrite(GPIO_PORTB_BASE, GPIO_PIN_7, 0xFF);
    
        delayPWM();
        
    GPIOPinWrite(GPIO_PORTB_BASE, GPIO_PIN_0, 0x00);            // GPIO Pins 0-7 auf Low stellen
    GPIOPinWrite(GPIO_PORTB_BASE, GPIO_PIN_1, 0x00);
    GPIOPinWrite(GPIO_PORTB_BASE, GPIO_PIN_2, 0x00);
    GPIOPinWrite(GPIO_PORTB_BASE, GPIO_PIN_3, 0x00);
    GPIOPinWrite(GPIO_PORTB_BASE, GPIO_PIN_4, 0x00);
    GPIOPinWrite(GPIO_PORTB_BASE, GPIO_PIN_5, 0x00);
    GPIOPinWrite(GPIO_PORTB_BASE, GPIO_PIN_6, 0x00);
    GPIOPinWrite(GPIO_PORTB_BASE, GPIO_PIN_7, 0x00);

   }
    return 0;
}
\end{lstlisting}